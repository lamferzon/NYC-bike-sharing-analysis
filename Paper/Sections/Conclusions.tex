\section{Conclusions}

At the end of our study we can conclude that the best spatio-temporal model to explain the bike sharing phenomenon is f-HDGM thanks to its capacity to work with data sampled in high frequency; its validation errors are more precise respect to the other models because they are built taking into account hourly data. However, the computational time requires to estimate it is greater than the other ones, therefore it is preferable to choose a not-functional statistical approach as the number of available data increases. To improve the performances of these models could be interesting to perform same the analyses on a dataset contained also bike sharing information regarding years prior the \num{2020}; in this way there would be the possibility of taking into consideration in models estimates also the periodicity of the phenomenon. Finally, weather variables are useful to forecast if  an individual will rent a bike, however they are not sufficient to explain our response variables completely due to social and individual reasons which are hard to get. 